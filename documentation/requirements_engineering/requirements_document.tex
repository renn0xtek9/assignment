\documentclass[a4paper,10pt]{article}
\usepackage{helvet}
\renewcommand{\familydefault}{\sfdefault}
\usepackage[utf8]{inputenc}
\usepackage{charter}
\usepackage[english]{babel}
\usepackage{amsmath}
\usepackage{amsfonts}
\usepackage{graphicx}
\usepackage{caption}
\usepackage{float}
\usepackage{hyperref}
\usepackage{setspace}
\usepackage[top=2cm,bottom=2cm,left=3cm,right=2cm]{geometry}
\usepackage[usenames,dvipsnames]{xcolor}
\usepackage{setspace}
\usepackage{fixltx2e}
\usepackage[style=numeric,backend=biber]{biblatex}
\usepackage[acronym]{glossaries}
\usepackage{makeidx}
\usepackage{nomencl}
\usepackage{textcomp}
\usepackage{sfmath}
\usepackage{booktabs}
\usepackage{algorithm}
\usepackage{algorithmic}
\usepackage{rotating}
\usepackage{listings}
\usepackage{bytefield}

% parameters
\newcommand{\parametertype}[5]{
\begin{center}
\begin{tabular}[c]{|c|c|c|c|c|}
\hline Type & Default value & Min & Max & Min Resolution \\
\hline #1 & #2 & #3 & #4 & #5\\
\hline
\end{tabular}
\end{center}
}

\newcommand{\parametercolor}[1]{{\color{blue}{#1}}}

\newcommand{\parameter}[1]{\parametercolor{\gls{#1}}}

\newcommand{\newparameter}[3]{
\newglossaryentry{#1}{
  type=parameters,
  name={#2},
  description={#3}
}
}
% usecases
\newcommand{\usecase}[3]{
\textbf{Usecase:} #1 \\
{\small ID: #2} \\
#3 \\}

% system requirements
\newcommand{\sysreq}[2]{\label{req-#1}\textbf{Sys-Req: #1} #2\newline\newline}
\newcommand{\linkreq}[1]{Req. \hyperref[req-#1]{#1}}

\makeatletter
\renewcommand{\maketitle}{
    \begin{titlepage}
        \newgeometry{left=2cm,right=2cm,top=2cm,bottom=2cm} % Set margins
        \noindent % Remove paragraph indentation
        \includegraphics[width=3cm]{./../resources/project_logo.png} % Insert logo on the left
        \hfill % Move the following text to the right
        \begin{minipage}[t]{0.6\textwidth} % Set width of the title text
            \vspace{0pt} % Align the top of the text with the logo
            \centering % Center the text horizontally
            \Huge\textbf{\MakeUppercase{\@title}}
            \\
            \vspace{10pt} % Add some vertical space between title and author
            \Large\textit{Deliverables description}\\ % Smaller, italic subtitle text
            \vspace{30pt} % Add some vertical space between author and date
            \vspace{10pt} % Add some vertical space between author and date
        \end{minipage}
        \vfill % Fill the vertical space between the title text and the bottom of the page
        \noindent % Remove paragraph indentation

        \centering{\textbf{\@author}}\\ % Large, bold author text
        \rule{\linewidth}{0.4pt} % Add a horizontal line across the page
        \vspace{5pt} % Add some vertical space between the line and the text below
        \footnotesize{\@date}
    \end{titlepage}
}
\makeatother


\include{acronym_glossary.tex}

\include{parameters.tex}

\include{title_format.tex}

\lstdefinestyle{cppstyle}{
    language=C++,
    basicstyle=\ttfamily\footnotesize, % Font style
    keywordstyle=\bfseries\color{blue}, % Keywords
    commentstyle=\itshape\color{gray}, % Comments
    stringstyle=\color{red}, % Strings
    numbers=left, % Line numbers on the left
    numberstyle=\tiny\color{gray}, % Line number style
    stepnumber=1, % Line number frequency
    numbersep=10pt, % Distance from line numbers to code
    tabsize=4, % Tab size
    showspaces=false, % Don't show spaces
    showstringspaces=false, % Don't show spaces in strings
    breaklines=true, % Automatic line breaking
    breakatwhitespace=true, % Only break at whitespace
    frame=single, % Adds a frame around the code
    rulecolor=\color{black}, % Frame color
    backgroundcolor=\color{lightgray!10}, % Background color
    morekeywords={uint8_t, uint16_t, uint32_t} % Additional keywords for C++
}


\title{Implementation of a driver for an Inertial Measurement Unit with UART communication protocol.}
\author{Maxime Haselbauer}
\addbibresource{./library.bib}
\setlength{\parindent}{0pt}
\makeindex
\makenomenclature
\makeglossaries

\begin{document}
\maketitle
\begin{abstract}
A draft of system and requirements analysis of an IMU driver communicating of UART.
\end{abstract}
\printnomenclature
\printglossaries

\section{Description of deliverables}
The deliverable contains:
\begin{itemize}
    \item The present document.
    \item A \texttt{flight-software-0.0.1-Release-Linux.deb} package containing release executables.
    \item A \texttt{flight-software-0.0.1-Release-Linux.deb} package containing debug executables together with extended code documentation for developer (doxygen, build process documentaition.)
\end{itemize}

The source code to generate those artifacts is hosted on \href{https://github.com/renn0xtek9/assignment}{github}.
The debian package shall be installable on a Linux Ubuntu 22.04 system.
It contains two executables:
\begin{itemize}
    \item \texttt{flight\_software}: an application that integrates the IMU Driver and display driver information and latest driver messages received.
    \item \texttt{software\_in\_the\_loop}: an application that simulate a fake IMU device and write data to virtual serial device file.
\end{itemize}
To setup the virtual device file the user must first run \texttt{setup\_software\_in\_the\_loop.sh} also provided within the package.
This shall setup the virtual serial device file \texttt{/tmp/vserial\_flight\_software\_side} and \texttt{/tmp/vserial\_sil\_side}. \newline
The user can then run \newline \texttt{flight\_software} \newline in a terminal and then \newline \texttt{software\_in\_the\_loop --duration\_ms=5000} \newline in another terminal. \\
This send some fake (constant) values in the virtual serial device file for 5 seconds.
The IMU message data will be displayed on the \texttt{flight\_software} terminal. Values should be nearly equal with respect to the LSB sensitivity.

\subsection{Code description}
The heart of the driver logic is located in the \href{https://github.com/renn0xtek9/assignment/blob/master/flight_software/libs/embedded_software/uart_imu_driver/src/driver.cpp}{\texttt{driver.cpp}} file.
The github repository shall be freely accessible. It is recommend to use the "Devcontainer" feature of Visual Studio Code if one want to create a developer environement.
Building instruction are provided in the \href{https://github.com/renn0xtek9/assignment/blob/master/CONTRIBUTING.md}{CONTRIBUTING.md} file.

\subsection{Driver architecture}
The driver is instatiated with:
\begin{itemize}
    \item a \texttt{OsAbstractionLayerInterface} object.
    \item a \texttt{DriverContext} object.
    \item an absolute path to the device file.
\end{itemize}

The OS abstraction layer abstracts all interaction with the actual operating system.
This is done for easy mocking and achieving 100\% code coverage (only line coverage, not branch coverage so far).

The driver context is the object that exchange data with the client application.
In an actual integration, this functionality is usually provided by the publisher/subscriber functionality of the middelware.\\
The driver has a \texttt{Start()} method that will spin a \texttt{Run()} in its own thread.
Calling the \text{Stop()} method shall stop it.\\
The conversion of message is done in the \href{https://github.com/renn0xtek9/assignment/blob/master/flight_software/libs/embedded_software/serializer/src/serializer.cpp}{serializer.cpp}.

The following rest of this documents is a non finalized analysis of the requirements and the design of the driver.
In a real context this work should order of magnitude more detailed.\\
I will be happy to discuss this work with you in a meeting.

\section{Introduction}
We will develop an driver for an IMU that communicate with our system over a serial communiction (UART).
For the pre-development, we make the following assumptions:
\begin{itemize}
    \item the IMU is a 6DOF sensor that provides acceleration and angular rates.
    \item the IMU does not store the current date and time internally.
    \item for the purpose of this work, we will develop the software on a Linux x86 platform. We assume that actual deployment will be on a POSIX compliant system.
    \item since the Linux Kernel already have a driver for UART perifpheral, the driver shall not implement low level functionality, like reassambling the bits in bytes.
    \item the IMU data collected by the driver are processed by an odometry application (that reconstruct the vehicle trajectory).
    We assume it is the only client of the IMU data.
    \item for simplicity we assume that the IMU is already configured to send data continuously at a fixed rate.
\end{itemize}

% file:///home/mhaselbauer/Downloads/lsm6ds3tr-c.pdf
We assume that the IMU has an acceleration range of $\pm 16g$ and angular rate range of $\pm 250^{\circ}/s$.
Both are represented by 16 bits values.
We assume the 16 bits are sent on the UART protocol in 2 bytes, with the least significant byte first.
We can obtain the acceleration and angular rates from their raw value coded on the 16 bits via.
\begin{equation}
    a = \frac{raw}{2^{15}} \times 16
\end{equation}
\begin{equation}
    \omega = \frac{raw}{2^{15}} \times 250
\end{equation}
We assume an output data rate of 208 Hz.

The temperature sensor measurement is stored on 16 bits with 256 LSB $^{\circ}C$.
\begin{equation}
    T^{\circ_C} = 25+\frac{raw}{256}
\end{equation}

\section{High level system overview}
This section is a superficial high level system design analysis. It addresses.
\begin{itemize}
    \item sequential interaction between the system subcomponents
    \item superficial fault tree analysis
    \item data structure and data flow
    \item software interface definitions
\end{itemize}
\textbf{The goal is to derive some of the most basic system requirements (see section: \ref{sec:system_requirements}).}

\subsection{Sequential interaction}
The system has following components:
\begin{itemize}
    \item the IMU itself.
    \item the UART peripheral of the compute platform.
    \item the operating system kernel.
    \item the device file that will be created by the kernel to describe the device.
    \item the IMU driver software component.
    \item the IMU message queue. This is typically a part of the middelware.
    \item the odometry application that will consume the IMU data.
\end{itemize}

\begin{figure}[H]
    \centering
    \includegraphics[width=1.0 \textwidth]{diagrams/high_level_sys_overview.png}
    \caption{High level sequence diagram of the nominal operation case}
    \label{fig-high-level-nominal}
\end{figure}

Fig.\ref{fig-high-level-nominal} describes the interaction sequence in the normal operation mode:\\
The IMU makes a new measurment at $t_{mes1}$.
It then sends a start byte to the UART to notify receiver of upcomming data frame. (See \linkreq{TS1}).
Once the first frame containing the start byte has been received, the low level UART driver of the kernel will write the byte to the device file.
Meanwhile the IMU driver is polling the kernel for the number of bytes available in the device file at high frequency. (See green lifeline.)
As soon as one byte is available, the IMU driver will read it and check if it is the start byte to detect the start of a new message (\linkreq{TS1}).
The IMU driver will then retrieve the clock time and use it to timestamp the measurement (\linkreq{5}).\\
Note that this approach has an intrinsic delay between the physical measurement and the timestamp.
Low-level drivers would perform better in this regard. In a real application it would be anyway mandatory to derive the maximum allowable difference from the requirements of the client (OdometryApp).\\
The IMU then idles for the duration of the transfer of the full message. (See blue lifeline).
The rest of the bytes are collected by the low-level driver.
The end of the IMU driver idling duration shall coincide with the last byte of the message being written in the device file.
The IMU driver then reads all the bytes and convert them to an IMU message (internal software representation).
The execution of the IMU driver shall not block the OdometryApp. Thus they must run in separate threads (\linkreq{6}).
To exchange data with the OdometryApp, the IMU driver shall store data in a shared memory queue (\linkreq{7}).\\
The memory will be locked during this process (See red lifeline).
This process is repeated several times until the OdometryApp purges all messages stored in the queue.\\
The OdometryApp will then process the IMU messages using their values and timestamps.
Therefore the driver must save timestamps together with each coresponding IMU messages (\linkreq{8}).

\subsection{Fault-tree analysis}
Because the IMU Driver is functional safety critical, it is important to derive the relevant technical safety requirements.
This section does not intend to create an exhaustive hazard and risk analysis with all possible failure modes.
Instead, it intends is to create a very superficial fault tree analysis to derive some of the most obvious safety relevant requirements.
We follow a top-down approach postulating a wrong trajectory reconstruction in the OdometryApp as the top event.
We categorize events as green, where mitigations measures shall arguably happen outside the IMU driver, and red where the IMU driver shall provide the mitigation.

\begin{figure}[H]
    \centering
    \includegraphics[width=1.0 \textwidth]{diagrams/main_fault_tree.drawio.png}
    \caption{Superficial trajectory reconstruction fault tree.}
    \label{fig-main-fault-tree}
\end{figure}

\begin{center}
\begin{tabular}{|p{5cm}|p{10cm}|}
\hline
\textbf{Failure mode} & \textbf{Mitigation} \\
\hline
IMU driver not deserializing data properly. & shall be addressed by rigorous unit testing of the implementation. \\
\hline
Message from IMU is incomplete. & shall be addressed by detecting the start of a message using a start byte \linkreq{TS1}. \\
\hline
The IMU baudrate is higher than expected & in this simple example, the mitigation consist in notifying the OdometryApp when it collects the IMU messages (\linkreq{TS2},\linkreq{TS3},\linkreq{TS4}). \\
\hline
The IMU message queue has reached max size & the queue shall not grow indefinitely to prevent memory allocation issue(\linkreq(TS8)). The mitigation for the possible drop of message consist in notifying with a counter of dropped message (\linkreq{TS5},\linkreq{TS6}). \\
\hline
The IMU has stop sending messages & the mitigation consist in setting the status to NO\_DATA (\linkreq{TS3},\linkreq{TS7}). \\
\hline
Delay between physical measure and timestamp & this lag will always be present, especially with our chosen implementation of a high-level, polling driver.
To mitigate this, we implement a polling mechanism that poll at higher frequency when expecting the arrival of a start byte to mitigate the delay(\linkreq{TS9}). \\
\hline
\end{tabular}
\end{center}

\subsection{Data structure and data flow}
We propose the following characteristics for the IMU. Those are inspired from \cite{lm6ds3}.
This is a MEMS IMU that is neither space grade nor UART capable, but provides good characeristics of a basic IMU.
The IMU will provide the following data:
\begin{itemize}
    \item $a_x$ acceleration on the x-axis.
    \item $a_y$ acceleration on the y-axis.
    \item $a_z$ acceleration on the z -axis.
    \item $\omega_x$ angular rate along its x-axis.
    \item $\omega_y$ angular rate along its y-axis.
    \item $\omega_z$ angular rate along its z-axis.
    \item $T$ temperature of the sensor.
\end{itemize}
From this we derive requirement \linkreq{1}.
We assume that the physical sampling of all those values are made at the exact same time, respectively the difference between the sampling time is negligible.
In other word we can timestamp the measurement time with a single timestamp $t_{mes}$.
According to the datasheets all values are represented by 16 bits values.
The temperature sensor has a sensitivity with 256 LSB $^{\circ}C$.
We assume the IMU would be configured with:
\begin{itemize}
    \item a range of $\pm 16g$ for the accelerometer. The sensitivity is $0.488~mg/LSB$
    \item a range of $\pm 250^{\circ}/s$ for the gyroscope. The sensitivity is $8.75~mdps/LSB$
    \item an output data rate of 208 Hz.
\end{itemize}

From this we derive \linkreq{2}, \linkreq{3} and \linkreq{4}.

\subsubsection{Definition of data frames and protocol.}
Every bytes is transmitted in a UART frame of 10 bits (1 start bit, 8 data bits, 1 stop bit).
\newline
\newline
\begin{bytefield}[bitwidth=4.1em]{10}
    \bitheader[endianness=little]{9,8,1,0}\\
    \bitbox{1}{Start bit}
    \bitbox{8}{Data bits}
    \bitbox{1}{Stop bit}
\end{bytefield}

The IMU transmit 1 start bytes and then 7 values of 2 bytes each (14 data bytes in total).
The complete message consist of 15 frames: 1 start frame (S) and 14 data frames (D1 to D14).
\newline
\newline
\begin{bytefield}[bitwidth=2.1em]{15}
    \bitheader[endianness=little]{14,1,0}\\
    \bitbox{1}{S}
    \bitbox{14}{D1 to D14}
\end{bytefield}

This amounts to 150 bits per message.
With an output data rate of 208 Hz, the required data rate is 31200 bit/s.
We choose a typical baudrate of 38400 bit/s.
This leads to a bit transmission duration of 26 $\mu$s.
The complete message (150 bits) transmission duration is 3906 $\mu$s.
With 208 Hz, the duration between the start of two messages is 4897 $\mu$s.
This leads to an approximative 900 $\mu$s of idle time between two messages.

\textbf{Note:} we will use those values in the code implemenation.
However, due to the non real time nature of a stock x86 Linux Ubuntu OS, the actual sleep duration will be inaccurate.
On our machine the minimum sleep duration is approx. 55 $\mu$s.
As mentionned earlier, this is a limit of a high-level polling approach.


\subsection{Software interface definition}
To fullfill \linkreq{1}, \linkreq{8} and \linkreq{NF1}, we define the following structure for IMU messages.
\begin{lstlisting}[style=cppstyle]
struct ImuData {
    float a_x{};
    float a_y{};
    float a_z{};
    float omega_x{};
    float omega_y{};
    float omega_z{};
    float temperature{};
    std::chrono::nanoseconds timestamp{};
};
\end{lstlisting}
\textbf{Note:} on x86 the float type is 32 bits long which is more than the strictly necessary 16 bits. We keep float type to be compliant with the C++17 standard library.
There are no requirements on the resolution of the timestamp. This shall be derived from the actual use cases of the OdometryApp.
For the sake of higher accuracy we choose nanoseconds.
Since the measurement would come at 208 Hz only we could potentially microseconds or even miliseconds.\\

To fulfill \linkreq{TS5} we propose the following structure.
\textbf{Note:} the queue will not fullfill \linkreq{TS8}. One could use a custom implementation later on, or this could be provided by a middelware.
\begin{lstlisting}[style=cppstyle]
struct ImuDataQueue {
  std::uint32_t dropped_message{0};
  std::queue<ImuData> data_queue{};
};
\end{lstlisting}

To fulfull \linkreq{TS2}, \linkreq{TS3}, \linkreq{TS7},
\begin{lstlisting}[style=cppstyle]
enum class ImuDriverStatus { OK, BUSY, NO_DATA };
\end{lstlisting}

\section{System requirements}
\sysreq{1}{The system shall represent each IMU measurements as a structure containing $a_x$, $a_y$, $a_z$, $\omega_x$, $\omega_y$, $\omega_z$ and $T$.}
\sysreq{2}{The system shall convert an array of 16 bit with the least significant byte first to an accleration in $g$.}
\sysreq{3}{The system shall convert an array of 16 bit with the least significant byte first to an angular rate in $\circ/s$.}
\sysreq{4}{The system shall convert an array of 16 bit with the least significant byte first to an temperature in $\circ C$.}
\sysreq{5}{The IMU driver shall retrieve the current time when the first byte of a new IMU mesurement is received by the UART peripheral.}
\sysreq{6}{The IMU driver shall be executed in a thread separated from the main application thread.}
\sysreq{7}{The IMU driver shall be write new IMU messages into a shared memory queue.}
\sysreq{8}{The IMU driver shall save the timestamp of every IMU message together with the message itself.}

\section{Validation and Verification}
\subsection{Software in the loop}
The valid the IMU driver, we develop a second application that will simulate the IMU device by sending fake measurement on a virtual serial device.
The test procedure for the IMU driver is as follow:
\begin{itemize}
    \item The tester allocate two device files to simulate virtual serial port.
    \item The tester starts the software in the loop application. The application immediately starts to write fake imu measurements data to the virtual device file.
    \item The tester starts the application that contains the driver. The driver application reads data from the virtual device file.
    \item The test application process the IMU message and determine if the test is a success or failure depending on the content.
\end{itemize}



\begin{figure}[ht]
    \centering
    \includegraphics[width=0.75 \textwidth]{diagrams/software_in_the_loop.png}
    \caption{High level overview of SIL test procedure}
    \label{reference}
\end{figure}



\printbibliography
\end{document}
