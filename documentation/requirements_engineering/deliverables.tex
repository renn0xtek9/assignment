\section{Description of deliverables}
The deliverable are:
\begin{itemize}
    \item The present document.
    \item A \texttt{flight-software-0.0.1-Release-Linux.deb} package containing release executables.
    \item A \texttt{flight-software-0.0.1-Release-Linux.deb} package containing debug executables together with code documentation for developers (doxygen, build process documentaition.)
\end{itemize}

The source code to generate those artifacts is hosted on \href{https://github.com/renn0xtek9/assignment}{github}.
The debian package shall be installable on a Linux Ubuntu 22.04 system.
It contains two executables:
\begin{itemize}
    \item \texttt{flight\_software}: an application that integrates the IMU Driver and displays driver information and the latest driver message received.
    \item \texttt{software\_in\_the\_loop}: an application that simulates a fake IMU device and writes data to a virtual serial device file.
\end{itemize}
To setup the virtual device file the user must first run \texttt{setup\_software\_in\_the\_loop.sh} also provided within the package.\\
This shall setup the virtual serial device file \texttt{/tmp/vserial\_flight\_software\_side} and \texttt{/tmp/vserial\_sil\_side}. \newline
The user can then run \newline \texttt{flight\_software} \newline in a terminal and then \newline \texttt{software\_in\_the\_loop --duration\_ms=5000} \newline in another terminal. \\
This sends some fake (constant) values to the virtual serial device file for 5 seconds.
The IMU message data will be displayed on the \texttt{flight\_software} terminal. Values should be nearly equal with respect to the LSB sensitivity.

\subsection{Code description}
The core of the driver logic is in the \href{https://github.com/renn0xtek9/assignment/blob/master/flight_software/libs/embedded_software/uart_imu_driver/src/driver.cpp}{\texttt{driver.cpp}} file.
It essentially implements a simple high-level driver that reads bytes from a device file using a polling mechanism.
This approach has been chosen for simplicity.
It has shortcommings in term of real time performance especially with respect of accuracy of the time stamping of the messages.
The github repository shall be freely accessible. It is recommend to use the "Devcontainer" feature of Visual Studio Code if one want to create a developer environement.
Building instruction are provided in the \href{https://github.com/renn0xtek9/assignment/blob/master/CONTRIBUTING.md}{CONTRIBUTING.md} file.

\subsection{Driver architecture}
The driver is instatiated with:
\begin{itemize}
    \item a \texttt{OsAbstractionLayerInterface} object.
    \item a \texttt{DriverContext} object.
    \item an absolute path to the device file.
\end{itemize}

The OS abstraction layer abstracts all interactions with the actual operating system.
This is done for easy mocking and achieving 100\% code coverage (only line coverage, not branch coverage so far).

The driver context is the object that exchange data with the client application.
In an actual integration, this functionality is usually provided by the publisher/subscriber functionality of the middelware.\\
The driver has a \texttt{Start()} method that will spin a \texttt{Run()} in its own thread.
Calling the \text{Stop()} method shall stop it.\\
The conversion of message is done in the \href{https://github.com/renn0xtek9/assignment/blob/master/flight_software/libs/embedded_software/serializer/src/serializer.cpp}{serializer.cpp}.

The rest of this documents is a very brief and non finalized analysis of the requirements and the design of the driver.
In a real context this work should order of magnitude more detailed.\\
I will be happy to discuss this work with you in a meeting.
